% ------------------------------------------------------------- 
% Arquivo :  relatório modelo                                        
% ------------------------------------------------------------- 
% O percentual(%) serve para incluir comentários: 
% tudo o que fica à direita dele não é interpretado pelo LaTex
% Linhas e espaços em branco também **NÃO** são
% interpretadas pelo LaTex

%% As intruções seguintes são o cabeçalho e devem estar antes do
%% \begin{document}

%\documenclass: mandatorio, indica o tipo/formato de documento
\documentclass[brazilian,12pt,a4paper,final]{article}
% tamanhos de fontes: 10pt, 11pt ou 12pt
% opções de estilo (padrões): article, report, book, slide, letter (artigo, relatorio, livro, apresentação de slides, carta)

%% Pacotes extras (opcionais):

% *babel* contem as regras de ifenização
\usepackage[brazil]{babel}

% *t1enc* permite o reconhecimento dos acentos inseridos com o teclado
%\usepackage{t1enc}

% *inputenc* com opção *utf8* permite reconhecimento dos caracteres com codificação UTF8, que é padrão dos esditores de texto no Linux. Isso permite reconhecimento automático de acentuação.
\usepackage[utf8]{inputenc}


% *graphicx* é para incluir figuras em formato eps 
\usepackage{graphicx} % para produzir PDF diretamente reescrever esta linha assim: \usepackage[pdftex]{graphicx}

% *color* fontes soloridas
\usepackage{color}
%%% fim do cabecalho %%%

\pagestyle{empty}
\title{Radiação de corpo negro}
\author{Aluno: Átila Leites Romero \\ Matrícula: 144679 \\ IF-UFRGS}

\begin{document}
\maketitle
\begin{abstract}
Verificação experimental da relação entre a temperatura de um corpo e a radiação por ele emitida.
\end{abstract}

\section{Introdução} 
%Pequeno histórico do problema. 
%Explicar porque o trabalho é relevante.


\section{Método}
% Aqui o Método 
%Detalhes sobre o método utilizado \cite{Kauffman_book},
%procure ao final do texto a referencia a esta bibliografia.
%demonstrações de porque ele funciona.  Limites analíticos, etc.
Uma lâmpada incandescente foi conectada a uma fonte regulável e instalada em frente a um sensor luminoso, conectado a um amplificador. Conforme a tensão aplicada à lâmpada era ajustada, mediu-se a corrente na lâmpada e a tensão resultante no sensor luminoso.

Admitindo que a radiação percebida pelo sensor seja proporcional à tensão de saída e que a temperatura do filamento seja proporcional à potência elétrica, a ... prevê que a tensão de saída do sensor deve ser proporcional à potência elétrica aplicada à lampada elevada a 4.

A primeira série de medidas, exibida na listagem a seguir,

\begin{table}[h]
\begin{tabular}{||l|l|l||} \hline
v(V)& i(A) & R(mV) & P(W)\\
\hline
 0 & 1 & 3\\
 1 & 2 & 4\\
 2 & 3 & 5\\
 \hline
 \end{tabular}
 \caption{A tabela mostra os valores de tempo, posiçao e velocidade do
{\ldots} }
\end{table}



Exemplo de fórmula matemática:
\mbox{}

$$ \int_{0}^{\infty} f(x) dx $$

\vspace{0.5cm}

Exemplo de lista numerada:
\begin{enumerate}
  \item{ primeiro }
  \item{ etc }
  \item{ etc }
\end{enumerate}

% O verbatim faz o latex ignorar a formatação: ai SIM os espaços e
% linhas em branco contam
% É util para trechos de código Fortran por exemplo
\vspace{0.3cm}
Exemplo de texto sem formatação para código {\bf FORTRAN} por exemplo
Veja o %\bf acima

\begin{verbatim}
...
Read (*,*) a, b, t

 Do i=0,t
    b(i) = a*c(i)
 End do
...

\end{verbatim}

\section{Resultados}
\vspace{2cm}
 Aqui os resultados, sua interpretação.

% Aqui incluimos uma figura
% Para testar este comando devem criar uma figura em formato
% postscript encapsulado (eps) e deve ter o mesmo nome que aparece entre chaves
% no caso ``fig.eps'' 
Incluindo uma figura em formato {\it eps}

\begin{figure}[hbtp]
\begin{center}
\includegraphics[width=8cm]{figura.eps}
\caption{Coloque aqui as legendas}
\label{fig}
\end{center}
\end{figure}
\vspace{0.5cm}

\begin{figure}[htbp]
\begin{center}
\rotatebox{-90}{\resizebox{8.0cm}{!}{\includegraphics{figura.eps}}}
\caption{Legendas}
\label{fig_rotacao}
\end{center}
\end{figure}



Incluindo uma tabela:

\section{Conclusões}
Recolocar resumidamente o problema, os resultados, as comparações \cite{Wolfram_book} com outros
trabalhos e as perspectivas futuras que o trabalho abre.

{\color{red} Este é um modelo geral, quando for utilizá-lo para um trabalho
específico leve em consideração as necesidades desse trabalho,
cuidando de omitir ou comentar com \% \% as seções que não
se apliquem.}

\begin{thebibliography}{99}

\bibitem{Kauffman_book}
S.~Kauffman, {\em The Origins of Order: Self-Organisation and
Selection in Evolution}, (Oxford University Press, 1993).

\bibitem{Wolfram_book}
S.~Wolfram, {\em Theory and Application of Cellular Automata},
(World Scientific, Singapore, 1986).

\end{thebibliography}

\end{document}

