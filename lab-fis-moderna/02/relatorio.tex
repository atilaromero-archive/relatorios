% ------------------------------------------------------------- 
% Arquivo :  relatório modelo                                        
% ------------------------------------------------------------- 

\documentclass[brazilian,12pt,a4paper,final]{article}
% tamanhos de fontes: 10pt, 11pt ou 12pt
% opções de estilo (padrões): article, report, book, slide, letter (artigo, relatorio, livro, apresentação de slides, carta)
\usepackage[brazil]{babel}     % ifenização
%\usepackage{t1enc}            % reconhecimento dos acentos inseridos com o teclado
\usepackage[utf8]{inputenc}    %  reconhecimento dos caracteres com codificação UTF8, acentuação.
\usepackage{graphicx}          % figuras em formato eps 
%\usepackage[pdftex]{graphicx} % para produzir PDF diretamente
%\usepackage{color}             % fontes soloridas
%%% fim do cabecalho %%%

\pagestyle{empty}
\title{Análise de dados medidos em um filamento de tungstênio}
\author{Aluno: Átila Leites Romero \\ Matrícula: 144679 \\ IF-UFRGS}

\begin{document}
\maketitle

\begin{abstract}
Este trabalho apresenta uma verificação experimental da teoria de radiação de corpo negro, 
utilizando montagem onde foi medida a radiação emitida por uma lâmpada de tungstênio 
em função da potência elétrica fornecida.
\end{abstract}

\section{Introdução}

\section{Procedimento experimental}

\section{Análise dos dados}

\section{Resultados}

\section{Discussão}

\section{Conclusões}


\begin{thebibliography}{99}
\end{thebibliography}

\end{document}

