% ------------------------------------------------------------- 
% Arquivo :  relatório modelo                                        
% ------------------------------------------------------------- 

\documentclass[brazilian,12pt,a4paper,final]{article}
% tamanhos de fontes: 10pt, 11pt ou 12pt
% opções de estilo (padrões): article, report, book, slide, letter (artigo, relatorio, livro, apresentação de slides, carta)
\usepackage[brazil]{babel}     % ifenização
%\usepackage{t1enc}            % reconhecimento dos acentos inseridos com o teclado
\usepackage[utf8]{inputenc}    %  reconhecimento dos caracteres com codificação UTF8, acentuação.
\usepackage{graphicx}          % figuras em formato eps 
%\usepackage[pdftex]{graphicx} % para produzir PDF diretamente
%\usepackage{color}             % fontes soloridas
%%% fim do cabecalho %%%

\pagestyle{empty}
\title{Análise de dados medidos em um filamento de tungstênio}
\author{Aluno: Átila Leites Romero \\ Matrícula: 144679 \\ IF-UFRGS}

\begin{document}
\maketitle

\begin{abstract}
Este trabalho apresenta uma verificação experimental da teoria de radiação de corpo negro, 
utilizando uma montagem onde foi medida a radiação emitida por uma lâmpada de tungstênio 
em função da potência elétrica fornecida.
\end{abstract}

\section{Introdução}
Segundo a lei de Stefan-Boltzmann, $$ R_{(T)}=\sigma T^4$$, onde $$R$$ é a potência total irradiada, $$\sigma$$ é uma constante e $$T$$ é a temperatura do corpo negro. 

Mas a emissividade de corpos reais é menor que a emissividade de um corpo negro ideal.
Por isso, para corpos reais, a equação é reescrita como $$ R_{(T)}=\epsilon(T)\sigmaT^4$$, onde $$\epsilon$$
é um número menor que $$1$$ e representa a emissividade do corpo.

Em outra experiência, foi verificado que as expressões
$$R=R_0+R_1(T-T_0)+R_2(T-T_0)^2$$ e 
$$T=T_0(\frac{r}{r_0)^\frac{1}{\gamma}$$, 
fornecem uma boa aproximação para a
relação entre resistência e temperatura do filamento de tungstênio.

A potência total dissipada por efeito Joule pode ser descrita por $$P=VI$$. 
Assumindo que a energia dissipada por condução e convecção varie lineramente 
com a temperatura, pode-se afirmar que $$P_D=D(T-T_0)$$.

Já a potência dissipada por radiação pode ser descrita pela lei de Stefan-Boltzmann, 
logo $$P=D(T-T_0)+S(T-T_0)$$, onde $$S=\sigma A 4\pi\epsilon$$.

Como $$\sigma$$ é muito pequeno, 
para baixas temperaturas a dissipação por condução e convecção prevalece e,
por isso, $$P\simeq D(T-T_0)$$ e $$(T-T_0)\simeq \frac{P}{D}$$,
o que leva a $$R=R_0+R_1\frac{P}{D}+R_2(\frac{P}{D})^2$$.

Para altas temperaturas, a potência irradiada passa a ser relevante, 

\section{Procedimento experimental}
Uma lâmpada de tungstênio com 20W de potência nominal foi ligada a uma fonte regulável. 
Um fotosensor foi instalado em frente à lâmpada e ligado a um amplificador de tensão.

Foram aplicados diferentes tensões à lâmpada.
Em cada etapa, eram medidas a corrente na lâmpada e a tensão de saída no fotosensor, já amplificada.
Assumiu-se que a luminância detectada seria proprocional à esta tensão de saída, mesmo sendo desconhecida 
a relação exata desta proporção.

A resistência da lâmpada pôde ser calculada para cada conjunto de dados através da lei de Ohm: $$ r=V/I $$, 
onde $$r$$ é a resistência, $$V$$ a voltagem e $$I$$ a corrente aplicada à lâmpada. 
A potência dissipada pela lâmpada foi calculada através de $$ r=VI $$.



\section{Análise dos dados}

\section{Resultados}

\section{Discussão}

\section{Conclusões}


\begin{thebibliography}{99}
\end{thebibliography}

\end{document}

