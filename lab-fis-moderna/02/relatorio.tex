% ------------------------------------------------------------- 
% Arquivo :  relatório modelo                                        
% ------------------------------------------------------------- 

\documentclass[brazilian,12pt,a4paper,final]{article}
% tamanhos de fontes: 10pt, 11pt ou 12pt
% opções de estilo (padrões): article, report, book, slide, letter (artigo, relatorio, livro, apresentação de slides, carta)
\usepackage[brazil]{babel}     % ifenização
%\usepackage{t1enc}            % reconhecimento dos acentos inseridos com o teclado
\usepackage[utf8]{inputenc}    %  reconhecimento dos caracteres com codificação UTF8, acentuação.
\usepackage{graphicx}          % figuras em formato eps 
%\usepackage[pdftex]{graphicx} % para produzir PDF diretamente
%\usepackage{color}             % fontes soloridas
%%% fim do cabecalho %%%

\pagestyle{empty}
\title{Análise de dados medidos em um filamento de tungstênio}
\author{Aluno: Átila Leites Romero \\ Matrícula: 144679 \\ IF-UFRGS}

\begin{document}
\maketitle

\begin{abstract}
Este trabalho apresenta uma verificação experimental da teoria de radiação de corpo negro, 
utilizando uma montagem onde foi medida a radiação emitida por uma lâmpada de tungstênio 
em função da potência elétrica fornecida.
\end{abstract}

\section{Introdução}
Segundo a lei de Stefan-Boltzmann, $$ R_{(T)}=\sigma T^4$$, onde $$R$$ é a potência total irradiada, $$\sigma$$ é uma constante e $$T$$ é a temperatura do corpo negro. 

Mas a emissividade de corpos reais é menor que a emissividade de um corpo negro ideal.
Por isso, para corpos reais, a equação é reescrita como $$ R_{(T)}=\epsilon(T)\sigmaT^4$$, onde $$\epsilon$$
é um número menor que $$1$$ e representa a emissividade do corpo.

Em outra experiência, foi verificado que as expressões
$$R=R_0+R_1(T-T_0)+R_2(T-T_0)^2$$ e 
$$R=R_0(\frac{T}{T_0)^\gamma$$, 
fornecem uma boa aproximação para a
relação entre resistência e temperatura do filamento de tungstênio.

A potência total dissipada por efeito Joule pode ser descrita por $$P=VI$$. 
Assumindo que a energia dissipada por condução e convecção varie lineramente 
com a temperatura, pode-se afirmar que $$P_D=D(T-T_0)$$.

Já a potência dissipada por radiação pode ser descrita pela lei de Stefan-Boltzmann, 
logo $$P=D(T-T_0)+S(T-T_0)$$, onde $$S=\sigma A 4\pi\epsilon$$.

Como $$\sigma$$ é muito pequeno, 
para baixas temperaturas a dissipação por condução e convecção prevalece e,
por isso, $$P\simeq D(T-T_0)$$ e $$(T-T_0)\simeq \frac{P}{D}$$,
o que leva a $$R=R_0+R_1\frac{P}{D}+R_2(\frac{P}{D})^2$$.

Para altas temperaturas, a potência irradiada passa a prevalecer, 
já que cresce muito mais rápido que a potência dissipada por difusão térmica.
Neste caso,  $$P\simeq S(T-T_0)$$ e, como $$T>>T_0$$, $$T^4\simeq \frac{P}{S}$$,
o que leva a $$R=R_0\frac{1}{T_0^\gama}(\frac{P}{S})^\frac{\gamma}{4}$$.

\section{Procedimento experimental}
Uma lâmpada de tungstênio com 20W de potência nominal foi ligada a uma fonte regulável. 
Um fotosensor foi instalado em frente à lâmpada e ligado a um amplificador de tensão.

Foram aplicados diferentes tensões à lâmpada.
Em cada etapa, eram medidas a corrente na lâmpada e a tensão de saída no fotosensor, já amplificada.
Assumiu-se que a luminância detectada seria proprocional à esta tensão de saída, mesmo sendo desconhecida 
a relação exata desta proporção.

A resistência da lâmpada pôde ser calculada para cada conjunto de dados através da lei de Ohm: $$ r=V/I $$, 
onde $$r$$ é a resistência, $$V$$ a voltagem e $$I$$ a corrente aplicada à lâmpada. 
A potência total dissipada pela lâmpada foi calculada através da lei de Joule, $$ P=VI $$.
\section{Análise dos dados}

O primeiro conjunto de dados 

\begin{table}[h]
\begin{tabular}{||r|r|r||} 
\hline
tempo& posição & velocidade\\
\hline
 0 & 1 & 3\\
 1 & 2 & 4\\
 2 & 3 & 5\\
V(V)    &       I(A)    &       L       &       R(Ohm)  &       P(W)    \\
0,11    &       0,12    &       5,7     &       0,92    &       0,01    \\
0,21    &       0,22    &       5,8     &       0,95    &       0,05    \\
0,30    &       0,27    &       5,9     &       1,11    &       0,08    \\
0,41    &       0,33    &       5,6     &       1,24    &       0,14    \\
0,50    &       0,35    &       6,0     &       1,43    &       0,18    \\
0,61    &       0,38    &       5,5     &       1,61    &       0,23    \\
0,70    &       0,40    &       6,0     &       1,75    &       0,28    \\
0,80    &       0,42    &       6,1     &       1,90    &       0,34    \\
0,90    &       0,44    &       5,1     &       2,05    &       0,40    \\
1,00    &       0,46    &       6,0     &       2,17    &       0,46    \\
1,10    &       0,48    &       6,0     &       2,29    &       0,53    \\
1,20    &       0,50    &       6,6     &       2,40    &       0,60    \\
1,29    &       0,52    &       6,9     &       2,48    &       0,67    \\
1,41    &       0,54    &       7,6     &       2,61    &       0,76    \\
1,50    &       0,55    &       8,7     &       2,73    &       0,83    \\
1,60    &       0,57    &       9,9     &       2,81    &       0,91    \\
1,68    &       0,59    &       11,4    &       2,85    &       0,99    \\
1,79    &       0,60    &       13,5    &       2,98    &       1,07    \\
1,88    &       0,62    &       15,7    &       3,03    &       1,17    \\
2,01    &       0,64    &       19,4    &       3,14    &       1,29    \\
2,53    &       0,72    &       44,7    &       3,51    &       1,82    \\
2,99    &       0,79    &       84,8    &       3,78    &       2,36    \\
3,46    &       0,85    &       146,2   &       4,07    &       2,94    \\
4,01    &       0,92    &       246,5   &       4,36    &       3,69    \\
4,55    &       0,98    &       370,8   &       4,64    &       4,46    \\
5,00    &       1,04    &       504     &       4,81    &       5,20    \\
5,53    &       1,09    &       689     &       5,07    &       6,03    \\
6,05    &       1,15    &       895     &       5,26    &       6,96    \\
7,02    &       1,25    &       1373    &       5,62    &       8,78    \\
8,01    &       1,34    &       1956    &       5,98    &       10,73   \\
9,00    &       1,43    &       2613    &       6,29    &       12,87   \\
10,05   &       1,52    &       3428    &       6,61    &       15,28   \\
11,02   &       1,60    &       4230    &       6,89    &       17,63   \\
12,08   &       1,69    &       5200    &       7,15    &       20,42   \\
13,04   &       1,76    &       6120    &       7,41    &       22,95   \\
 \hline
 \end{tabular}
 \caption{A tabela mostra os valores de tempo, posiçao e velocidade do
{\ldots} }
\end{table}



\section{Resultados}

\section{Discussão}

\section{Conclusões}


\begin{thebibliography}{99}
\end{thebibliography}

\end{document}

