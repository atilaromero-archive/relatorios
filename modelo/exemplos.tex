\begin{enumerate}
  \item{ primeiro }
\end{enumerate}

\mbox{}

$$ \int_{0}^{\infty} f(x) dx $$

\vspace{0.5cm}

Exemplo de texto sem formatação para código {\bf FORTRAN} por exemplo
Veja o %\bf acima
\begin{verbatim}
Read (*,*) a, b, t
 Do i=0,t
    b(i) = a*c(i)
 End do
\end{verbatim}

Incluindo uma figura em formato {\it eps}

\begin{figure}[hbtp]
\begin{center}
\includegraphics[width=8cm]{figura.eps}
\caption{Coloque aqui as legendas}
\label{fig}
\end{center}
\end{figure}
\vspace{0.5cm}

\begin{figure}[htbp]
\begin{center}
\rotatebox{-90}{\resizebox{8.0cm}{!}{\includegraphics{figura.eps}}}
\caption{Legendas}
\label{fig_rotacao}
\end{center}
\end{figure}

Incluindo uma tabela:
\begin{table}[h]
\begin{tabular}{||l|c|r||} \hline
tempo& posição & velocidade\\
\hline
 0 & 1 & 3\\
 1 & 2 & 4\\
 2 & 3 & 5\\
 \hline
 \end{tabular}
 \caption{A tabela mostra os valores de tempo, posiçao e velocidade do
{\ldots} }
\end{table}

{\color{red} Este é um modelo geral, quando for utilizá-lo para um trabalho
específico leve em consideração as necesidades desse trabalho,
cuidando de omitir ou comentar com \% \% as seções que não
se apliquem.}
Recolocar resumidamente o problema, os resultados, as comparações \cite{Wolfram_book} com outros
\begin{thebibliography}{99}
\bibitem{Kauffman_book}
S.~Kauffman, {\em The Origins of Order: Self-Organisation and
Selection in Evolution}, (Oxford University Press, 1993).

\bibitem{Wolfram_book}
S.~Wolfram, {\em Theory and Application of Cellular Automata},
(World Scientific, Singapore, 1986).
\end{thebibliography}

